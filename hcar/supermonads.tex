% Supermonads-JS.tex
\begin{hcarentry}{Supermonads}
\report{Jan Bracker}%05/17
\status{Experimental fully working version}
\participants{Jan Bracker and Henrik Nilsson}
\makeheader

The supermonad package provides a unified way to represent different monadic
and applicative notions. In other words, it provides a way to use standard and generalized
monads and applicative functors (with additional indices or constraints)
without having to manually disambiguate which notion is referred to in every context.
This allows the reuse of code, such as standard library functions, across all of 
the notions.

To achieve this, the library splits the monad and applicative type classes
such that they are general enough to allow instances for all of the generalized notions and
then aids constraint checking through a GHC plugin to ensure that everything
type checks properly. Due to the plugin the library can only be used with GHC.

If you are interested in using the library, we have a few examples of
different size in the repository to show how it can be utilized. The generated
Haddock documentation also has full coverage and can be seen on the libraries
Hackage page.

The project had its first release shortly before ICFP and the Haskell
Symposium 2016. Since then we have added support for applicative 
functors in addition to monads. The support for applicative 
functors has not been released on Hackage at the time of writing, 
but is available on GitHub.
We are working on a comprehensive paper that covers all aspects of the project
and its theoretical foundations.
It should be available within the next year.

If you are interested in contributing, found a bug or have a suggestion to
improve the project we are happy to hear from you in person, by email or over
the projects bug tracker on GitHub.

\FurtherReading
\begin{compactitem}
  \item Hackage:\\ \url{http://hackage.haskell.org/package/supermonad}
  \item Repository:\\ \url{https://github.com/jbracker/supermonad}
  \item Paper:\\
    \url{http://www.cs.nott.ac.uk/~psxjb5/publications/2016-BrackerNilsson-Supermonads.pdf}
  \item Bug-Tracker:\\ \url{https://github.com/jbracker/supermonad/issues}
  \item Haskell Symposium presentation:\\ \url{https://youtu.be/HRofw58sySw}
\end{compactitem}
\end{hcarentry}

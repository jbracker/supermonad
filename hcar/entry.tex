\documentclass[DIV16,twocolumn,10pt]{scrreprt}
\usepackage{paralist}
\usepackage{graphicx}
\usepackage[final]{hcar}

%include polycode.fmt

\begin{document}

\begin{hcarentry}{Supermonads}
\report{Jan Bracker}
\status{Experimental fully working version}
\participants{Jan Bracker and Henrik Nilsson}% optional
\makeheader

The supermonad package provides a unified way to represent different 
monadic notions. In other words, it provides a way to use standard and 
generalized monads (with additional indices or constraints)
with each other without having to manually disambiguate which notion 
is referred to in every computation. To achieve this, the library represents 
monads as a set of two type classes that are general enough to 
allow instances for all of the different notions and then aids 
constraint checking through a GHC plugin to ensure that everything type checks properly.
Due to the plugin the library can only be used with GHC.

If you are interested in using the library, we have a few examples of different
size in the repository to show how it can be utilized. The generated Haddock 
documentation also has full coverage and can be seen on the libraries 
Hackage page.

The project had its first release shortly before ICFP and the Haskell Symposium 2016.
We are currently working on providing the same kind of support for applicative
functors and arrows, so that generalizations of these notions can be 
used as freely as the different notions of monads.

If you are interested in contributing, found a bug or have a suggestion 
to improve the project we are happy to hear from 
you in person, by email or over the projects bug tracker on GitHub.

\FurtherReading
\begin{itemize}
  \item Hackage:\\ 
        \url{http://hackage.haskell.org/package/supermonad}
  \item Repository:\\ 
        \url{https://github.com/jbracker/supermonad}
  \item Paper:\\ 
        \url{http://www.cs.nott.ac.uk/~psxjb5/publications/2016-BrackerNilsson-Supermonads.pdf}
  \item Bug-Tracker:\\ 
        \url{https://github.com/jbracker/supermonad/issues}
  \item Haskell Symposium presentation:\\
        \url{https://youtu.be/HRofw58sySw}
\end{itemize}

\end{hcarentry}

\end{document}
